\documentclass[11pt]{article}
\usepackage{a4wide,parskip}
\usepackage{hyperref}
\usepackage{titlesec}

\titleformat{\section}{\normalfont\fontsize{12}{15}\bfseries}{\thesection}{1em}{}

\begin{document}

\centerline{\Large R209 Essay:  Usable security}
\vspace{2em}
\centerline{\large Chongyang Shi (\emph{cs940})}
\vspace{1em}
\centerline{\large \today}
\vspace{1em}

\section{Summaries of research}

In order to determine whether security softwares with user interfaces conventionally considered well-designed can adequately assist novice users in performing complex security tasks, Whitten and Tygar \cite{whitten1999johnny} performed a case study on PGP 5.0. They evaluated the security usability of PGP's UI through two methods: a cognitive walkthrough of potential confusions within the UI a novice user may face, and user tests conducted with potential users unfamiliar with the software and the PGP concept. Results from the evaluations showed that PGP 5.0's UI fails to adequately demonstrate the concepts of email encryption and authentication, and various problems in visual design can lead to confused users performing dangerously insecure operations. However, those critical of this study may consider the choice of PGP 5.0 to be sub-optimal for evaluations, as the protocol design of PGP is considerably more complex than other security tools a novice user may come across, such as symmetric file encryption with a pre-shared key \cite{gujrati2013usability}. This may have hindered performance in user tests.

Herley \cite{herley2014more} reviewed the state of contemporary usable security research, and argued that past efforts in making security components of softwares more usable may have been misled. The author argued that many past efforts have overestimated the capacity of a user completing security-related tasks, and could not produce a usable solution accomplishable within the limited time and attention the user could provide. Exaggeration of average-case security risk \cite[2.3]{herley2014more} and unrealistic burdening of security mandates \cite[1.3]{herley2014more} can also lead to the user cutting corners on security. Therefore, the author believes that future research on usable security should be efficiency-focused. While this paper effectively summarised the perceived shortcomings in usable security research, it is relatively thin on recommendations to overcome these problems.

Most recently, Glass et al. \cite{glass2016usability} studied how orderings and transitions between components of a security-related task could impact the usability of the task. Based on the real world scenario of using an airline self-service check-in kiosk, transitions between a set of partially-ordered tasks were modelled as a constraint satisfaction problem (CSP), whose optimal solution in ordering tasks largely concurred with the best user performance in user tests and the favoured solution by experts surveyed \cite[Fig. 7]{glass2016usability}. This verified the viability of the cognitive framework established by the authors. Some limitations of the models were recognised by the authors, the most critical of which is the potential lack of consistency with user expectations on how certain tasks are ordered, prompting future research.

\section{Key themes of research}

\subsection{Sell security to users on incentives, not on endless mandates}

A theme observed across all three papers is the need to provide users with good reasons to use security, rather than mandating the users with endless security policies which may result in workarounds and corner-cuttings being invented. Limitations and ineffectiveness of security automations and user trainings were observed by Whitten and Tygar \cite[Sec. 1]{whitten1999johnny}. Herley noted that security mandates may not convince the users that the cost of security compliance is outweighed by its benefits. Also noted is that some mandates such as periodic password change accomplish very little \cite[1.3, 2.2]{herley2014more}. Glass et al. \cite[Sec. VII]{glass2016usability} found that when the security mandate (in this case, passport authentication) is placed in the least obstructive way within procedures, users are most satisfied with the process.

\subsection{Provide the user with minimal information required for security}

It was observed by all three papers that when presented with an overwhelming amount of security information, the user tends to perform less well in accomplishing the task. During the PGP 5.0 cognitive analysis by Whitten and Tygar, too much information about PGP keys, automatically generated trustworthiness levels, and even excessive vocabulary could confuse a novice user \cite[4.6, 4.7]{whitten1999johnny}. Herley \cite[2.1]{herley2014more} believes that security advices have provided users with too much information to follow effectively. The cognitive framework of Glass et al. \cite[III. D.]{glass2016usability} also focuses on improving performance by minimising overall task demand.

\subsection{Simplify organisational goals to ease mandates on users}

A final theme covered in various ways by the three papers is the need for policymakers and software engineers to simplify their security goals, in order to reduce the amount of mandates on users. While under ideal circumstances users will sign PGP keys of those they trust, Whitten and Tygar \cite[4.4]{whitten1999johnny} believe that this is beyond the abilities of novice users, and hence trust levels should not be prominently displayed by PGP 5.0 by default to avoid confusing novice users. Herley \cite[2.4]{herley2014more} illustrated the example of avoid requiring users to choose passwords resilient against offline attacks by adequately protecting stored hashed passwords. By simplifying the organisational security goals without compromising a reasonable level of security, user experience with security procedures can be vastly improved.  

\section{Ideas of current context}

Evolving desktop operating systems have provided more flexibility in designing user interfaces and displaying metaphor icons \cite[4.1]{whitten1999johnny} since Whitten and Tygar's \cite{whitten1999johnny} evaluation of PGP in 1999. The theoretical basis of public key cryptography has also not changed much other than acceptable key sizes. However, modern PGP implementations still fail to achieve good usable security, as assessed by Sheng et al. \cite{sheng2006johnny} and Ruoti et al. \cite{ruoti2015johnny} in 2006 and 2015 respectively. Similar user tests were performed across all three studies of PGP. While improvements to the interface design were recommended and improved between studies, the fundamental problem of the PGP protocol being hard to understand by novice users persists \cite[p. 4]{ruoti2015johnny}, which needs to be addressed by improving user training.

While being relatively recent research, Herley's \cite{herley2014more} idea of security problems are often caused by users overloaded with security-related tasks rather than by insufficient security measures is now well-concurred. Efforts have been made to reduce the amount of security-related tasks users must perform everyday. These are represented in organisational settings by single sign-on (SSO) \cite[8.1]{sasse2014great}, and in mobile devices by the introduction of fingerprint unlocking \cite{cherapau2015impact} and more recently face-recognition unlocking \cite{applefaceid}.

An alternative interpretation of the persistent difficulties faced by novice users of PGP, supported by Herley's \cite{herley2014more} argument that difficult security mandates overwhelm users, is that certain security measures may be fundamentally beyond the target user group, due to the measures being inherently too complex. This has been studied with a complexity framework by Benenson et al. \cite{benenson2015maybe}. This could mean that the study of usable security will hit a roadblock without developing new security protocols that are more easily understood by novice users.

\section{Literature review}

In addition to the two further usability reviews of PGP implementations since 1999 \cite{sheng2006johnny, ruoti2015johnny}, there have been research efforts in dropping the knowledge-based authentication
systems all together (along with their flaws in usability and user knowledge), and replacing with recall-based authentication, achieving promising results \cite{dhamija2000deja}. However, significant shortcomings in recall-based authentication have been noted \cite{wiedenbeck2005passpoints}, making practical implementations difficult. 

As with the aforementioned complexity study \cite{benenson2015maybe}, Gerck \cite{gerck2007secure} also found that the technical complexity of PGP constrains security usability, and alternative protocol designs may be required. Further to Herley \cite{herley2014more}, discrepancies in security practices of expert and novice users have also been noted by Ion et al. \cite{ion2015no}. Some practical means of educating users about security devices such as PGP have been recommended by Redmiles et al. \cite{redmiles2016think}. 

A key component of the cognitive framework by Glass et al. \cite[III. A.]{glass2016usability} is the prediction of time required to complete a given task, which can be based on CogTool \cite{bellamy2011deploying}. CogTool is widely used in cognitive modelling, such as evaluating helicopter interfaces \cite{ludwig2006comparing} and keystroke on handheld devices \cite{luo2005predicting}. The framework's use of constraint satisfaction (CSP) in cognitive modelling has similarly been conducted to model air traffic controller behaviours \cite{cerone2005formal} and human in virtual environments \cite{smith1999using}.

\emph{(1212 words according to texcount.)}

\bibliographystyle{IEEEtran}
\footnotesize{\bibliography{week3}}


\end{document}
