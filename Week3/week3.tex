\documentclass[11pt]{article}
\usepackage{a4wide,parskip}
\usepackage{hyperref}
\usepackage{titlesec}

\titleformat{\section}{\normalfont\fontsize{12}{15}\bfseries}{\thesection}{1em}{}

\begin{document}

\centerline{\Large R209 Essay:  Usable security}
\vspace{2em}
\centerline{\large Chongyang Shi (\emph{cs940})}
\vspace{1em}
\centerline{\large \today}
\vspace{1em}

This essay provides a synthesis of three papers focused on usable security. While being a sub-field of human-computer interaction (HCI), usable security may present a very different set of challenges from broader user-centred design principles \cite[Abs.]{whitten1999johnny}. Two user-facing security systems were studied: user interface for PGP 5.0 \cite{whitten1999johnny} and the airline check-in kiosk \cite{glass2016usability}, in addition to a generalised study on the state of usable security \cite{herley2014more}.

\section{Summaries of research}

\section{Key themes of research}

\subsection{Sell security to users on incentives, not on endless mandates}

\subsection{Simplify organisation goals to ease mandates on users}

\subsection{Provide the user with minimal information required for security}

\section{Ideas of current context}

\section{Literature review}

% \emph{(1245 words according to texcount.)}

\bibliographystyle{IEEEtran}
\footnotesize{\bibliography{week3}}


\end{document}
